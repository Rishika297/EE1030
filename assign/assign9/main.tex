\iffalse
\chapter{2017}
\author{AI24BTECH11020}
\section{ae}
\fi

%\begin{enumerate}[start=1]
\item Given the vectors $\vec{v_1}=\hat{i}+3\hat{j};\vec{v_2}=2\hat{i}-4\hat{j}+3\hat{k},$ the vector $\vec{v_3}$ that is perpendicular to both $\vec{v_1}$ and $\vec{v_2}$ is given by:
		\begin{enumerate}
	\item $v_3=v_1-\brak{v_1\cdot v_2}\frac{v_2}{\norm{v_2}}$
	\item $v_3=\hat{k}$
	\item $v_3=v_2-\brak{v_1\cdot v_2}\frac{v_1}{\norm{v_1}}$
	\item $v_3=\frac{v_1\times v_2}{\norm{v_1\times v_2}}$
		\end{enumerate}
\item The value of the integral $I=\int_c\brak{\brak{x-y}dx+x^2dy}$, with $C$ the boundary of the square $0\leq x\leq 2;0\leq y\leq 2,$ is \underline{\hspace{2cm}}

\item Let $\vec{v}\brak{t}$ be a unit vector that is a function of the parameter $t$. Then $\vec{v}\cdot\frac{d\vec{v}}{dt}=$ \underline{\hspace{2cm}}

\item The eigenvalues $\lambda_n$ and eigenfunctions $u_n\brak{x}$ of the sturm-Liouville problem $$\frac{d^2y}{dx^2}+k^2\lambda y=0, 0<x<1; y\brak{0}=0,y\brak{1}=0$$\\
		are given by:
 \begin{enumerate}
	 \item $\lambda_n=n^2\pi^2;u_n\brak{x}=\sin\lambda_nx,   n=0,\pm1,\pm2,\cdots,\infty$
	 \item $\lambda_n=\frac{n^2\pi^2}{k^2};u_n\brak{x}=\sin kn\pi x,   n=0,\pm1,\pm2,\cdots,\infty$
	 \item $\lambda_n=\frac{n^2\pi^2}{k^2};u_n\brak{x}=\sin n\pi x,   n=0,\pm1,\pm2,\cdots,\infty$
	 \item $\lambda_n=n^2\pi^2;u_n\brak{x}=\sin n\pi x,   n=0,\pm1,\pm2,\cdots,\infty$
 \end{enumerate}

\item $3$-point Gaussian integration formula is given by:
	$\int_{-1}^{1}f\brak{x}dx \approx \Sigma_{j=1}^{3} A_jf\brak{x_j}$ with $x_1=0,x_2=-x_3=-\sqrt{\frac{3}{5}};A_1=\frac{8}{9},A_2=A_3=\frac{5}{9}.$
		This formula exactly integrates
		\begin{enumerate}
			\item $f\brak{x}=5-x^7$
			\item $f\brak{x}=2+3x+6x^4$
			\item $f\brak{x}=13+6x^3+x^6$
			\item $f\brak{x}=e^{-x^2}$
		\end{enumerate}

\item Which one of the following statements is NOT true
	\begin{enumerate}
        \item The pitching moment of any airfoil at any angle of attack is always zero at the center of pressure
        \item The pitching moment of any airfoil at any angle of attack is always zero at the aerodynamic center
        \item The center of pressure and aerodynamic center coincide for a symmetric airfoil
        \item The pitching moment about the aerodynamic center,for any airfoil, does not vary with angle of attack
	\end{enumerate}

\item Which one of the following statements is NOT true
	\begin{enumerate}
		\item Compared to a laminar boundary layer, a turbulent boundary layer is more desirable on a wing operating at large angle of attack
		\item The skin friction drag for a turbulent boundary layer is larger than that for a laminar boundary layer
		\item The location of transition from laminar to turbulent boundary layer depends only on the operating Reynolds number
		\item A seperated flow does not necessarily lead to a turbulent boundary layer
	\end{enumerate}
\item A De Laval nozzel is to be designed for an exit Mach number of $1.5$. The reservoir conditions are given as $P_{\circ}=1 atm(gage),T_{\circ}=20^{\degree}C,\gamma=1.4.$ Assuming shock free flow in the nozzle, the exit absolute pressure(in atm) is \underline{\hspace{2cm}} (in three decimal places)

\item Consider a steady one dimensional flow of a perfect gas with heat transfer in a duct. The T-s diagram (shown below) shows both the static and the stagnation conditions at two locations, A and B,in the duct. $A_t$ and $B_t$ denote stagnation conditions for states $A$ and $B$, respectively. It is known that $\brak{\Delta T}_A=\brak{\Delta T}_B. M_A$ and $M_B$ are the Mach numbers of the flow at locations $A$ and $B$.\\
	\begin{center}
\begin{tikzpicture}
\draw[->] (0,0)--(0,4.5);\node at (0,5){T};
\draw[->] (0,0)--(6,0);\node at (6.2,0){S};
\draw[dotted] (0,4)--(4.5,4)--(4.5,3)--(0,3);
\draw[dotted] (0,2)--(1.5,2)--(1.5,1)--(0,1);
\draw(1,0.9)parabola(5.2,4);
\node at (1.5,0.6){$A$};
\node at (4.7,2.8){$B$};
\node at (1.5,2.2){$A_t$};
\node at (4.7,4.2){$B_t$};
\draw[<->](-0.2,3)--(-0.2,4);
\draw[<->](-0.2,1)--(-0.2,2);
\fill[black] (1.5,1) circle (2pt);
\fill[black] (4.5,3) circle (2pt);
\draw (3.4,1.7)--(3.5,2)--(3.2,1.9);
\node at (-0.8,3.5){$(\Delta T)_B$};
\node at (-0.8,1.5){$(\Delta T)_A$};
\end{tikzpicture}
\end{center}
Which of the following statements is true about the flow.
\begin{enumerate}
	\item Flow is subsonic and $M_A<M_B$
	\item Flow is supersonic and $M_A>M_B$
	\item Flow is subsonic and $M_A>M_B$
	\item Flow is supersonic and $M_A<M_B$
\end{enumerate}

\item To ensure only the longitudinal static stability (and not the condition for equilibrium) of a low speed aircraft, the aircraft components must be designed to satisfy which one of the following conditions:
	\begin{enumerate}
		\item $\frac{\partial C_m}{\partial \alpha}<0$ and $C_{m_0}>0$
		\item $\frac{\partial C_m}{\partial \alpha}<0$ 
		\item $\frac{\partial C_m}{\partial {C_L}}<0$ and $C_{m_0}<0$
		\item $\frac{\partial C_m}{\partial {C_L}}=0.0$
	\end{enumerate}

\item Which of the following statement(s) is/are true about the shear centre of a cross-section:\\
	P: It is that point in the cross-section through which shear loads produce no twisting.\\
	Q: This point is also the centre of twist of sections subjected to pure torsion.\\
	R: The normal stress at this point is always zero.\\
	\begin{enumerate}
		\item $P,Q$ and $R$
		\item $P$ only
		\item $P$ and $Q$ only
		\item $P$ and $R$ only
	\end{enumerate}

\item Let $\overline{N_m}$ and $\overline{N_0}$ be respectively the non-dimensional locations of the stick-fixed maneuver point and stick-fixed neutral point of a low speed convectional aircraft. These distances are measured with respect to the nose of the fuselage. The numerical value of $\overline{N_m}-\overline{N_0}$
	\begin{enumerate}
		\item will always be negative
		\item will always be positive
		\item will always be zero
		\item can have any value depending on the location of the center of gravity of the aircraft
	\end{enumerate}
\item The phenomenon of rudder lock in conventional low speed aircraft is primarily due to 
	\begin{enumerate}
		\item large value of directional derivative, $C_{n\beta}$
		\item the sidewash due to fuselage on the vertical stabilizer
		\item the tendency of rudder to float rapidly at high angles of side-slip the sidewash due to wing on the vertical stabilizer 
	\end{enumerate}
%\end{enumerate}


