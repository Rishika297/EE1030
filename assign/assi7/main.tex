%iffalse
\let\negmedspace\undefined
\let\negthickspace\undefined
\documentclass[journal,12pt,onecolumn]{IEEEtran}
\usepackage{cite}
\usepackage{amsmath,amssymb,amsfonts,amsthm}
\usepackage{algorithmic}
\usepackage{graphicx}
\usepackage{textcomp}
\usepackage{xcolor}
\usepackage{txfonts}
\usepackage{listings}
\usepackage{enumitem}
\usepackage{mathtools}
\usepackage{gensymb}
\usepackage{comment}
\usepackage{multicol}
\usepackage[breaklinks=true]{hyperref}
\usepackage{tkz-euclide}
\usepackage{listings}
\usepackage{gvv}
%\def\inputGnumericTable{}                                 
\usepackage[latin1]{inputenc}
\usepackage{color}
\usepackage{array}
\usepackage{longtable}
\usepackage{calc}
\usepackage{multirow}
\usepackage{hhline}
\usepackage{ifthen}
\usepackage{lscape}
\usepackage{tabularx}
\usepackage{array}
\usepackage{float}
\usepackage{circuitikz}

\newtheorem{theorem}{Theorem}[section]
\newtheorem{problem}{Problem}
\newtheorem{proposition}{Proposition}[section]
\newtheorem{lemma}{Lemma}[section]
\newtheorem{corollary}[theorem]{Corollary}
\newtheorem{example}{Example}[section]
\newtheorem{definition}[problem]{Definition}
\newcommand{\BEQA}{\begin{eqnarray}}
\newcommand{\EEQA}{\end{eqnarray}}
\renewcommand{\define}{\stackrel{\triangle}{=}}
\theoremstyle{remark}
\newtheorem{remark}{Remark}

% Marks the beginning of the document
\begin{document}
\bibliographystyle{IEEEtran}
\vspace{3cm}

\title{ph-$2011$-$14$ to $26$}
\author{AI24BTECH11020 - Rishika}
\maketitle
\bigskip
\renewcommand{\thefigure}{\theenumi}
\renewcommand{\thetable}{\theenumi}
\begin{enumerate}[start=14]
\item If $L_x,L_y$ and $L_z$ are respectively the $x,y$ and $z$ components of angular momentum operator $L$, the commutator $\sbrak{L_xL_y,L_z}$ is equal to
                \begin{enumerate}
        \item $\iota\hbar\brak{L_x^2+L_y^2}$
        \item $2\iota\hbar L_z$
        \item $\iota\hbar\brak{L_x^2-L_y^2}$
        \item $0$
                \end{enumerate}
\item The normalized ground state wavefunction of a hydrogen atom is given by $\psi\brak{r}=\frac{1}{\sqrt{4\pi}}\frac{2}{a^{\frac{3}{2}}}e^{\frac{-r}{a}}$, where $a$ is the Bohr radius and $r$ is the distance of the electron from the nucleus,located at the origin.The expectation value $\left\langle \frac{1}{r^2} \right\rangle$ is
                \begin{enumerate}
                        \item $\frac{8\pi}{a^2}$
                        \item $\frac{4\pi}{a^2}$
                        \item $\frac{4}{a^2}$
                        \item $\frac{2}{a^2}$
                \end{enumerate}
\item Two charges $q$ and $2q$ are placed along the $x$-axis in front of a grounded, infinite conducting plane, as shown in the figure. They are located respectively at a distance of $0.5 m$ and $1.5 m$ from the plane. The force acting on the charge $q$ is\\
        \begin{center}
        \begin{tikzpicture}
                \fill[black] (-0.05,-2) rectangle (0,2);
                \draw[thick] (0,-1.9)--(-0.5,-1.9);
                \draw[thick] (-0.5,-1.9)--(-0.5,-2.1);
                \draw[thick](-0.8,-2.1)--(-0.2,-2.1);
                \draw[thick](-0.7,-2.2)--(-0.3,-2.2);
                \draw[thick](-0.6,-2.3)--(-0.4,-2.3);
    \node[circle, fill=black, inner sep=1pt, label=above:\(q\)] at (2,0) {};
    \node[circle, fill=black, inner sep=1pt, label=above:\(2q\)] at (4,0) {};
    \draw[->] (0,0) -- (5,0) node[right] {\(x\)};
    \draw[<-] (0, -0.3) -- (0.4,-0.3) node[right] {0.5 m};
                \draw[->](1.6,-0.3)--(2,-0.3);
    \draw[<-] (0, -0.6) --(1.4,-0.6) node[right]{1.5 m};
    \draw[->](2.8,-0.6)--(4,-0.6);
        \end{tikzpicture}
                \begin{enumerate}
        \item $\frac{1}{4\pi\varepsilon}\frac{7q^2}{2}$
        \item $\frac{1}{4\pi\varepsilon}2q^2$
        \item $\frac{1}{4\pi\varepsilon}q^2$
        \item $\frac{1}{4\pi\varepsilon}\frac{q^2}{2}$
                \end{enumerate}
\end{center}
\item A uniform surface current is flowing in the positive $y$-direction over an infinite sheet lying in $x-y$ plane. The direction of the magnetic field is
        \begin{enumerate}
\item along $\hat{i}$ for $z>0$ and along $\hat{-i}$ for $z<0$
\item along $\hat{k}$ for $z>0$ and along $\hat{-k}$ for $z<0$
\item along $\hat{-i}$ for $z>0$ and along $\hat{i}$ for $z<0$
\item along $\hat{-k}$ for $z>0$ and along $\hat{k}$ for $z<0$
        \end{enumerate}
\item A magnetic dipole of dipole moment $\vec{m}$ is placed in a non-uniform magnetic field $\vec{B}$. If the position vector of the dipoleis $\vec{r}$,the torque acting on the dipole about the origin is
        \begin{enumerate}
\item $\vec{r}\times\brak{\vec{m}\times\vec{B}}$
\item $\vec{r}\times \nabla \brak{\vec{m}\cdot\vec{B}}$
\item $\vec{m}\times\vec{B}$
\item $\vec{m}\times\vec{B}+\vec{r}\times\nabla\brak{\vec{m}\cdot\vec{B}}$
        \end{enumerate}
\item Which of the following expressions for a vector potential $\vec{A}$ DOES NOT represent a uniform magnetic field of magnitude $B_0$ along the $z$-direction ?
        \begin{enumerate}
                \item $\vec{A}=\brak{0,B_0x,0}$
                \item $\vec{A}=\brak{-B_0y,0,0}$
                \item $\vec{A}=\brak{\frac{B_0x}{2},\frac{B_0y}{2},0}$
                \item $\vec{A}=\brak{-\frac{B_0y}{2},\frac{B_0x}{2},0}$
        \end{enumerate}
\item A neutron passing through a detector is detected because of
        \begin{enumerate}
\item the ionization it produces
\item the scintillation light it produces
\item the electron - hole pairs it produces
\item the secondary particles produced in a nuclear reaction in the detector medium
        \end{enumerate}
\item An atom with one outer electron having orbital angular momentum $l$ is placed in a weak magnetic field. The number of energy levels into which the higher total angular momentum state splits, is
        \begin{enumerate}
                \item $2l+2$
                \item $2l+1$
                \item $2l$
                \item $2l-1$
        \end{enumerate}
\item For a multi-electron atom,$l,L$ and $S$ specify the one-electron orbital angular momentum,total orbital angular momentum and total spin angular momentum, respectively. The selection rules for electric dipole transition between the two electronic energy levels, specified by $l,L$ and $S$ are
        \begin{enumerate}
\item $\Delta L=0,\pm1;\Delta S=0;\Delta l=0,\pm1$
\item $\Delta L=0,\pm1;\Delta S=0;\Delta l=\pm1$
\item $\Delta L=0,\pm1;\Delta S=\pm1;\Delta l=0,\pm1$
\item $\Delta L=0,\pm1;\Delta S=\pm1;\Delta l=\pm1$
        \end{enumerate}
\item For a three-dimensional crystal having $N$ primitive unit cells with a basis of $p$ atoms, the number of optical branches is
        \begin{enumerate}
                \item $3$
                \item $3p$
                \item $3p-3$
                \item $3N-3p$
        \end{enumerate}
\item For an intrinsic semiconductor, $m_e^*$ and $m_h^*$ are respectively the effective masses of electrons and holes near the corresponding band edges. At a finite temperature, the position of the Fermi level
        \begin{enumerate}
\item depends on $m_e^*$ but not on $m_h^*$
\item depends on $m_h^*$ but not on $m_e^*$
\item depends on both $m_e^*$ and $m_h^*$
\item depends neither on $m_e^*$ nor on $m_h^*$
        \end{enumerate}
\item In the following circuit, the voltage across and the current through the $2k\Omega$ resistance are
\begin{center}
\begin{circuitikz}
    \draw (0,0)--(0,1.9);\draw (0,2.1)--(0,4); \node at (1.1,2){30 $V$};\node at (0.4,2.3) {+};
    \draw (-0.4,1.9)--(0.4,1.9);\draw (-0.6,2.1)--(0.6,2.1);
    % Resistors in series
    \draw (0,4) -- (1,4)
    to[R=500~$\Omega$] (3,4)
    to[R=1~k$\Omega$] (5,4) -- (7,4)
    to [R=2~k$\Omega$] (7,0)--(5,0);
    \draw (0,0) -- (3,0)--(5,0);
    \draw (3,0)--(3,1.5);\draw (2.5,1.5)--(3.5,1.5);
    \draw (2.5,1.5)--(3,2);\draw (3.5,1.5) -- (3,2); \draw (2.5,2) -- (3.5,2); \draw (2.3,1.8) -- (2.5,2); \draw (3.7,2.2) -- (3.5,2);\draw (3,2)--(3,4);\node at (2.6,2.2) {$20 V$} ;
    \draw (5,0)--(5,1.5);\draw (4.5,1.5)--(5.5,1.5);
    \draw (4.5,1.5)--(5,2);\draw (5.5,1.5) -- (5,2);\draw (4.5,2) --(5.5,2);\draw (4.3,1.8) --(4.5,2);\draw (5.7,2.3) --(5.5,2); \draw (5,2) --(5,4);\node at (4.6,2.2) {$10 V$};
\end{circuitikz}
\end{center}
\begin{enumerate}
    \item $20 V,10 mA$
    \item $20 V,5 mA$
    \item $10 V,10 mA$
    \item $10 V,5 mA$
\end{enumerate}
\section{$Q.26$ to $Q.55$ carry two marks each.}
\item The unit vector normal to the surface $x^2+y^2-z=1$ at the point $P\brak{1,1,1}$ is 
\begin{enumerate}
    \item $\frac{\hat{i}+\hat{j}-\hat{k}}{\sqrt{3}}$
    \item $\frac{2\hat{i}+\hat{j}-\hat{k}}{\sqrt{6}}$
    \item $\frac{\hat{i}+2\hat{j}-\hat{k}}{\sqrt{6}}$
    \item $\frac{2\hat{i}+2\hat{j}-\hat{k}}{3}$
\end{enumerate}

\end{enumerate}
\end{document}



~
~

