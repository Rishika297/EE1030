%iffalse
\let\negmedspace\undefined
\let\negthickspace\undefined
\documentclass[journal,12pt,onecolumn]{IEEEtran}
\usepackage{cite}
\usepackage{amsmath,amssymb,amsfonts,amsthm}
\usepackage{algorithmic}
\usepackage{graphicx}
\usepackage{textcomp}
\usepackage{xcolor}
\usepackage{txfonts}
\usepackage{listings}
\usepackage{enumitem}
\usepackage{mathtools}
\usepackage{gensymb}
\usepackage{comment}
\usepackage{multicol}
\usepackage[breaklinks=true]{hyperref}
\usepackage{tkz-euclide} 
\usepackage{listings}
\usepackage{gvv}                                        
%\def\inputGnumericTable{}                                 
\usepackage[latin1]{inputenc}                                
\usepackage{color}                                            
\usepackage{array}                                            
\usepackage{longtable}                                       
\usepackage{calc}                                             
\usepackage{multirow}                                         
\usepackage{hhline}                                           
\usepackage{ifthen}                                           
\usepackage{lscape}
\usepackage{tabularx}
\usepackage{array}
\usepackage{float}
\usepackage{circuitikz}

\newtheorem{theorem}{Theorem}[section]
\newtheorem{problem}{Problem}
\newtheorem{proposition}{Proposition}[section]
\newtheorem{lemma}{Lemma}[section]
\newtheorem{corollary}[theorem]{Corollary}
\newtheorem{example}{Example}[section]
\newtheorem{definition}[problem]{Definition}
\newcommand{\BEQA}{\begin{eqnarray}}
\newcommand{\EEQA}{\end{eqnarray}}
\renewcommand{\define}{\stackrel{\triangle}{=}}
\theoremstyle{remark}
\newtheorem{remark}{Remark}

% Marks the beginning of the document
\begin{document}
\bibliographystyle{IEEEtran}
\vspace{3cm}

\title{ae-$2017$-$1$ to $13$}
\author{AI24BTECH11020 - Rishika}
\maketitle
\bigskip
\renewcommand{\thefigure}{\theenumi}
\renewcommand{\thetable}{\theenumi}
\begin{enumerate}
	\item Once the team of analysts identify the problem, we \underline{\hspace{2cm}} in a better position to comment on the issue.\\Which one of the following choices CANNOT fill the given blank?
		\begin{enumerate}
			\item will be
			\item were to be
			\item are going to be 
			\item might be
		\end{enumerate}
	\item A final examination is the \underline{\hspace{2cm}} of a series of evaluations that a student has to go through.
		\begin{enumerate}                           
                        \item culmination                
                        \item consultaion               
                        \item desperation
                        \item insinuation
                \end{enumerate}

	\item If IMHO=JNIP;IDK=JEL; and SO=TP, then IDC=\underline{\hspace{2cm}}.
		\begin{enumerate}                           
                        \item JDE                  
                     \item JED
                        \item JDC
                        \item JCD
                \end{enumerate}
	\item The product of three integers $X,Y$ and $Z$ is $192$. $Z$ is equal to $4$ and $P$ is equal to the average of $X$ and $Y$. What is the minimum possible value of $P$?
		\begin{enumerate}                           
                        \item $6$                          
                        \item $7$                        
                        \item $8$
                        \item $9.5$
                \end{enumerate}

	\item Are there enough seats here? There are \underline{\hspace{2cm}} people here than I expected.
		\begin{enumerate}                           
                        \item many                          
                        \item most                          
                        \item least
                        \item more
                \end{enumerate}

	\item Fiscal deficit was $4\%$ of the GDP in $2015$ and that increased to $5\%$ in $2016$. If the GDP increased by $10\%$ from $2015$ to $2016$, the percentage increase in the actual fiscal deficit is $\underline{\hspace{2cm}}$.
		\begin{enumerate}                           
                        \item $37.50$                      
                        \item $35.70$               
                        \item $25.00$
                        \item $10.00$
                \end{enumerate}

	\item Two pipes $P$ and $Q$ can fill a tank in $6$ hours and $9$ hours respectively, while a third pipe $R$ can empty the tank in $12$ hours. Initially, $P$ and $R$ are open for $4$ hours. Then $P$ is closed and $Q$ is opened. After $6$ more hours $R$ is closed. The total time taken to fill the tank (in hours) is \underline{\hspace{2cm}}.
		\begin{enumerate}                           
                        \item $13.50$                  
                        \item $14.50$                 
                        \item $15.50$
                        \item $16.50$
		\end{enumerate}
	\item While teaching a creative writing class in India, I was surprised at receiving stories from the students that were all set in distant places; in the American West with cowboys and in Manhattan penthouses with clinking ice cubes. This was, till an eminent Caribbean writer gave the writers in the once-colonised countries the cofidence to see the shabby lives around them as worthy of being "told"\\ The writer of this passage is surprised by the creative writing assignments of his students, because \underline{\hspace{2cm}}.
		\begin{enumerate}
			\item Some of the students had written stories set in foreign places
			\item None of the students had written stories set in India
			\item None of the students had written about ice cubes and cowboys
			\item Some of the students had written about ice cubes and cowboys
		\end{enumerate}
	\item Mola is a digital platform for taxis in a city. It offers three types of rides - Pool, Mini and Prime. The table below presents the number of rides for the past four months. The platform earns one US dollar per ride. What is the percentage share of revenue contributed by Prime to the total revenues of Mola, for the entire duration?
\begin{table}[h!]
\centering
\begin{tabular}[12pt]{ |c| c| c|}
    \hline
	parameter & Description & value \\ 
    \hline
	 C & Centre & $\myvec{0\\ 13/6}$\\
    \hline 
	 O & point1 & $\myvec{0\\0}$\\
    \hline
	 P & point2 & $\myvec{2\\3}$\\
    \hline   
	 r & radius & 13/6\\
    \hline
    \end{tabular}

\end{table}
\begin{enumerate}
	\item $16.24$
	\item $23.97$
	\item $25.86$
	\item $38.74$
\end{enumerate}
\item $X$ is an online media provider. By offering unlimited and exclusive online content at attractive prices for a loyalty membership, $X$ is almost forcing its customers towards its loyalty membership. If its loyalty membership continues to grow at its current rate, within the next eight yeats more households will be watching $X$ than cable telivision.\\ Which one of the following statements can be inferred from the above paragraph?
	\begin{enumerate}
		\item Most households that subscribe to $X$'s loyalty membership discontinue watching cable television
		\item Non-members perfer to wwatch cable television
		\item Cable television operators don't subscribe to $X$'s loyalty membership
		\item The X is cancelling accounts of non-members
	\end{enumerate}
\item Let $X$ be the Poisson random variable with parameter $\lambda=1$. Then, the probability $P\brak{2\leq X\leq 4}$ equals
	\begin{enumerate}
		\item $\frac{19}{24e}$
		\item $\frac{17}{24e}$
		\item $\frac{13}{24e}$
		\item $\frac{11}{24e}$
\end{enumerate}
\item For the series $\sum_{n=1}^{\infty}\frac{\brak{x+1}^n}{n 2^n},-\infty<x<\infty,$ which of the following statements is NOT correct?
	\begin{enumerate}
	\item The series converges at x=-3
	\item The series converges at x=-1		  \item The series converges at x=0
	\item The series converges at x=1
\end{enumerate}
\item Let $f\brak{z}= \overline{z}e^{-\abs{z}^2}$, wher $\overline{z}$ is the complex conjugate of $z$. Then, it is differentiable on
	\begin{enumerate}
		\item $\abs{z}>1$
		\item $\abs{z}<1$
		\item $\abs{z}=1$
		\item the entire complex plane $\mathbb{C}$
	\end{enumerate}
\end{enumerate}
\end{document}

