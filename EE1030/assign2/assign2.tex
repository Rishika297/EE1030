\let\negmedspace\undefined
\let\negthickspace\undefined
\documentclass[journal]{IEEEtran}
\usepackage[a5paper, margin=10mm, onecolumn]{geometry}
%\usepackage{lmodern} % Ensure lmodern is loaded for pdflatex
\usepackage{tfrupee} % Include tfrupee package

\setlength{\headheight}{1cm} % Set the height of the header box
\setlength{\headsep}{0mm}     % Set the distance between the header box and the top of the text

\usepackage{gvv-book}
\usepackage{gvv}
\usepackage{cite}
\usepackage{amsmath,amssymb,amsfonts,amsthm}
\usepackage{algorithmic}
\usepackage{graphicx}
\usepackage{textcomp}
\usepackage{xcolor}
\usepackage{txfonts}
\usepackage{listings}
\usepackage{enumitem}
\usepackage{mathtools}
\usepackage{gensymb}
\usepackage{comment}
\usepackage[breaklinks=true]{hyperref}
\usepackage{tkz-euclide} 
\usepackage{listings}
% \usepackage{gvv}                                        
\def\inputGnumericTable{}                                 
\usepackage[latin1]{inputenc}                                
\usepackage{color}                                            
\usepackage{array}                                            
\usepackage{longtable}                                       
\usepackage{calc}                                             
\usepackage{multirow}                                         
\usepackage{hhline}                                           
\usepackage{ifthen}                                           
\usepackage{lscape}
\begin{document}

\bibliographystyle{IEEEtran}
\vspace{3cm}

\title{1.1.2.13}
\author{AI24BTECH11020 - RISHIKA KOTHA}
% \maketitle
% \newpage
% \bigskip
{\let\newpage\relax\maketitle}

\renewcommand{\thefigure}{\theenumi}
\renewcommand{\thetable}{\theenumi}
\setlength{\intextsep}{10pt} % Space between text and floats


\numberwithin{equation}{enumi}
\numberwithin{figure}{enumi}
\renewcommand{\thetable}{\theenumi}

\textbf{Question}\\
The fourth vertex $\vec{D}$ of a parallelogram ABCD whose three vertices are $\vec{A}(-2, 3),\vec{B}(6, 7)$ and $\vec{C}(8, 3)$ is
\\
\textbf{Solution}\\
Given,
\begin{align}
\vec{A}=\myvec{-2\\3}\\
\vec{B}=\myvec{6\\7}\\
\vec{C}=\myvec{8\\3}\\
\end{align}
To find the vertex D of a parallelogram,
\begin{align}
	\vec{D}=\vec{A}+\vec{C}-\vec{B}\\
        \vec{D}=\myvec{-2+8-6\\3+3-6}\\
	\vec{D}=\myvec{0\\-1}\\
\end{align}
therefore,the coordinates of the fourth vertex $\vec{D}$ are $\myvec{0\\-1}$\\
\end{document}
